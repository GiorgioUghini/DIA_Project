As this experiment is substantially the union of the budget and the pricing problem, and given that the problem can be decomposed, no new Learner is implmented here. Instead, a new Environment is created, as this time it should keep memory of the real behaviour of the users based on the budget allocated and of the real behaviour of the users based on the pricing chosen. The core of the experiment here is the set up of the two classes of Learners (GPTS Learner of the budgeting and the TS Learner for the pricing) in a way that they can work together.

Aside from the common variables \textit{TIME-SPAN}, \textit{N-CLASSES} and \textit{N-EXP}, in the main function we had to set up the parameter under the control of the company, that are:

\begin{itemize}
	\item \textbf{Budget bounds}: the variables min-budget and max-budget are arrays that contains the budget bount for the budget problem. In this case, the company has chosen to allocate a minimum budget of \EUR{1000} to each subcampaign while it can afford to spend more on its top class buyers, respectively \EUR{5400} units for class 1, \EUR{5800} for class 2 and \EUR{5200} for class 3.
	\\\item \textbf{Pricing bounds}: with the algorithm we designed, it will be possible for the company to fix a different price for each category of users so it makes sense to give the possibility to the company to set different price bounds for each class of users. In this experiment, the company set the same bounds for all the classes \EUR{100}-\EUR{400} but the possibility to change is still possible by setting the parameters of the algorithm.
	\item \textbf{Maxmium budget}: this is the costraint needed for the knapsack-like problem that represent the maximum amount of money the company wants to spend. For this experiment this variable is set to \EUR{9000}.
	\item \textbf{Granularity}: as for the first experiment (the budgeting problem alone), the allocation of budget has a granularity of \EUR{200}. With the same granularity but different allocation range, each GP Learner resulted in a different number of arms (respectively 22 arms for North/WithChildren, 24 arms for North/WithoutChildren and 21 arms for South/WithChildren).
\end{itemize}