As we said in the first section, the product taken into analysis is a seasonal item. Using Google Trends, we were able to identify the region of maximum and minimum interest of the users regarding a Louis Vuitton scarf. In the graph below is clear that this product is searched primarily in the cold seasons and it is year after year more demanded by the customers. In the chart the x-axis represents a five year time period while the y-axis represents the popularity of the keywords "Louis Vuitton Scarf" searched on Google (100 is the moment of maximum interest while 0 is the minimum).
\makebox[\textwidth][c]{\includegraphics[width=1.2\textwidth]{sections/images/trend}}
Based on the diagram we decided to divide the year in four periods of time, each one underlining a different phase.\newline\\
\makebox[\textwidth][c]{\includegraphics[width=0.8\textwidth]{sections/images/months}}
In order to better understand our reasoning behind the decisions that brought us to choose as graphs for the four different phases the ones showed below, is important to notice some remarks. \\In the first phase (Jan-Feb-Mar) the growth of the curve is almost linear, since there are no competitors that can interfere with our advertising of the product, but still there is a large pool of users to influence, since it is a phase of high interest of the product.\\ In the second phase (Apr-May-Jun) instead, since in these months the interest about the item is low, there will be less people interest in the scarf, therefore the impact that a higher budget allocated to the ads can make is less relevant that the impact that it can produce on a high interest phase. Based on this assumption we decided that the graph for the second phase should have a more remarkable growth with a small budget allocated to the ads, settling to much more slower growth as the budget significantly increases. \\In the next phase (Jul-Aug-Sep), when the new competitor enters the market, the graph should be different in the low budget allocated range respect to the second graph, while maintaining a similar profile in the high budget allocated range. The cause is the following: since we need to beat the competition in order to sell our product, we will struggle to do so allocating a small amount of money to the advertisement, while instead allocating a sufficient amount of money, we should be able to attract user to us as if there was not a competitor in the market. \\Finally, in the last phase (Oct-Nov-Dec) we focussed on the fact that since it is an high interest phase, but this time with a competitor that challenges us, the budget that needs to be allocated to the sub-campaigns still needs to be a considerable amount if we want to attract a sufficient number of users as in the third phase, but in this period the pool of users that can be reached is much wider, translating in a more linear curve in the high budget allocated range with respect to the third phase.\\
Taken into consideration our reasoning, the images below will present the structure of the probability distribution over the daily number of clicks for every value of budget allocated to that sub-campaign.\\
\makebox[\textwidth][c]{\includegraphics[width=0.85\textwidth]{../curves/real_curves/daily_clicks_10}} 
\makebox[\textwidth][c]{\includegraphics[width=0.85\textwidth]{../curves/real_curves/daily_clicks_00}} 
\makebox[\textwidth][c]{\includegraphics[width=0.85\textwidth]{../curves/real_curves/daily_clicks_01}} 
\newline\\\\
\makebox[\textwidth][c]{\includegraphics[width=0.85\textwidth]{../curves/real_curves/daily_clicks_11}} 