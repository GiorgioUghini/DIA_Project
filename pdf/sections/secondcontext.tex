After having analized the budget allocation over the three subcampaigns considering only one phase, we are moving to a more general context.
\\In particular, we supposed the existance of four different phases in a year, that are:
\begin{itemize}
	\item \textbf{JAN-FEB-MAR}: These months can still be considered "cold months", so the \emph{interest} in a scarf \emph{is still high}. In addition, we suppose that \emph{no competitor} will launch a new product in these months because that competitor would miss the OCT-NOV-DEC high interest phase.
	\item \textbf{APR-MAY-JUN}: In these spring months \emph{the interest} in a scarf \emph{becomes low} as temperatures arises. Of course, \emph{no competitors} are present for the same motivation as above.
	\item \textbf{JUL-AUG-SEP}: If a \emph{new competitor wants to join the market}, the best time of the year to do so is the late-summer/beginning of fall. In this way, even if this phase can still be considered a \emph{low interest phase}, the competitor could start to promote its product to be ready for late-fall and winter.
	\item \textbf{OCT-NOV-DEC}: A \emph{new competitor has joined the market} in the previous months and now the customer base has to take a decision on which is their favourite, possibly based on the marketing strategy each party encompasses. These cold months will be, of course, a \emph{high interest phase}.
\end{itemize}