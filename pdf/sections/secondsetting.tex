In case of non-stationary environment, it is important to define a
proper value for the sliding window, that is, the length of the buffer where the latest
collected samples are stored. We considered two different formulas to compute a
reasonable sliding window length, which pros and cons are reported below:

\begin{itemize}
	\item \textbf{Unknown environment}:\@  This is the basic formula for the computation of the window length. It only considers the number of interactions T our algorithm is going to perform
	with the environment.
	$$SW_1 = \sqrt{T}$$
	It is a very simple formula, altough with large numbers of arms it leads to bad performances, as it does not take them into account.
	
	\item \textbf{Some environment knowelge}:\@  Provided that all the phases of the demand curve have the same length so that they are uniformly distributed over the time horizon, we’re setting the sliding window size equal to a third of this length, in order to be more ready in case of a abrupt change (like in correspondence of the beginning
	of a new phase) at the cost of re-exploring the same arms at least 2 times inside the same phase. With $P$ as the number of phases:
	$$SW_2 = \dfrac{T}{3P}$$
	This is a very simple formula, tailored for a specific case, the phases should be uniformly distributed over the time horizon
	and their lengths have to be equal, which is not always the case.	
\end{itemize} 

We supposed that the compay could have a small prior information about the distribution of the phase that could come from previous campaign analisys or other sources.
\\As the phases are uniformly distributed and we know the number of it and the optimization horizon, we implemented the second formula for the Sliding Window lenght that resulted in:
$$SW = \dfrac{365}{4*3} = 1 \text{ month}$$
This is a reasonable amount of time as, provided the existance of 4 phases in a year, each phase would last 3 months and using this $SW$ time we're sure that, after a month, a new phase has begun and no more dirty data are considered.
In addition, the parameters that we set are as follows:
$budgets_{min} = [10, 10, 10]$, $budgets_{max} = [70, 80, 60]$, $step = 2$ and $budget_{tot} = 110$.